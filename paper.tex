\documentclass[10pt, conference, compsocconf]{IEEEtran}

% packages
\usepackage{algorithm}
\usepackage{algorithmic}
\usepackage{amsfonts} % for R symbol (the set of real numbers)
\usepackage{color}
\usepackage[pdftex]{graphicx}
\usepackage{graphicx}
\usepackage[bookmarks=false]{hyperref}
\hypersetup{colorlinks=true,linkcolor=black,citecolor=black,filecolor=black,urlcolor=blue}
\usepackage{mathtools}
\usepackage{multirow}
\usepackage{stmaryrd} % for llbracket and rrbracket
\usepackage{subcaption}
\usepackage{nicefrac}
\usepackage{amsmath}
\DeclarePairedDelimiter{\ceil}{\lceil}{\rceil}
\DeclarePairedDelimiter{\floor}{\lfloor}{\rfloor}

% new commands
\newcommand{\todo}[1]{\marginpar{\parbox{18mm}{\flushleft\tiny\color{red}\textbf{TODO}:
      #1}}}
\newcommand{\note}[1]{
  \color{blue}\emph{[Note: #1]}
  \color{black}
}

\begin{document}

\title{A multi-dimensional extension of the Lightweight Temporary Compression method}

\author{Bo Li, Tristan Glatard\\
  Department of Computer Science and Software Engineering\\ Concordia University, Montreal, Quebec, Canada\\
  {first.last}@concordia.ca \vspace*{-0.5cm}}

\maketitle

\begin{abstract}
\end{abstract}

\section{Introduction}

LTC is among the stream compression method that provides the highest 
compression rate for the lowest resource (CPU, RAM) consumption. As 
such, it makes it a very good candidate for the compression of data 
streams acquired in embedded, low-power systems such as connected 
objects, systems on module (SoMs), etc The type of data acquired on 
such systems, however, is often multi-dimensional. For instance 
accelerometers and gyroscopes usually measure variables along 3 
directions, these variables being distinct but related through the 
actual movement of the sensor. In this paper, we investigate the 
extension of LTC to higher dimensions. First, we express the algorithm 
in an arbitrary vectorial space of dimension $n$. Second, we 
instantiate it for the Euclidean and xxx distances, in spaces of 
dimension 2D+t and 3D+t. Finally, we compare the multi-dimensional LTC 
to the 1-dimensional version, on gyroscopic data acquired on a SoM 
device.

\section{Lightweight Temporal Compression}

Summarize existing paper.

\section{Extension to dimension $n$}

\section{Implementation}

Our code is available at ... (make a clean release!)

\section{Results}

\subsection{Data sets}

\subsection{Compression ratios}

\subsection{Errors}

\subsection{Memory consumption}

\section{Conclusion}

\section*{Acknowledgement}


\bibliographystyle{IEEEtran}
\bibliography{IEEEabrv,biblio.bib}


\end{document}
